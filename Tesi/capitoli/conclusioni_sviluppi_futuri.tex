\chapter{Conclusioni e Sviluppi Futuri} %\label{1cap:spinta_laterale}
% [titolo ridotto se non ci dovesse stare] {titolo completo}
%


\begin{citazione}
	\textit{In questo capitolo vengono riassunti i risultati ottenuti in questo lavoro di Tesi e si presentano al contempo alcune idee per un possibile sviluppo futuro della piattaforma.}
\end{citazione}

\newpage

\section{Conclusioni}
	Con questo lavoro di Tesi si è voluto proporre una possibile soluzione ai problemi che affliggono gli utenti del trasporto pubblico locale nella loro esperienza quotidiana. Si è parlato di come l'affollamento e l'irregolarità che, purtroppo, molto spesso caratterizzano le corse in momenti difficili come quello che si sta affrontando in questo periodo rappresentino un discreto disagio per i fruitori del servizio. In particolare, con questa prima \textit{milestone} dell'applicazione si è voluto dare la priorità al creare un'infrastruttura funzionante con le features più importanti già presenti. Tra queste, riassumendo a grandi linee, si trovano: 
	\begin{itemize}
		\item Segnalazione e verifica di fermate, linee ed aziende di trasporto;
		\item Consultazione delle segnalazioni sopracitate;
		\item Tracciamento in diretta delle corse espletate dalle diverse linee.
	\end{itemize}

	\section{Sviluppi Futuri}
		Allo stato attuale, \textsc{ShallWeGo} non è pronta ad essere inserita in un ambiente di \textit{produzione}. Al momento, sono stati individuati diversi possibili sviluppi che potrebbero portare la piattaforma ad uscire dallo stato di demo ed essere rilasciata al pubblico. Nello specifico: 

		\begin{itemize}
			\item Dare la possibilità ad un utente di apportare modifiche ad una segnalazione perché non aggiornata o con informazioni incomplete (come ad esempio la presenza di quadri orari affissi presso una fermata successivamente alla data di segnalazione della stessa);
			\item Specificare, nell'ambito della segnalazione di una linea, l'ordine in cui si succedono le fermate che quest'ultima serve con le sue corse nell'arco di una giornata. Questo permetterebbe di mostrare efficientemente la prossima fermata di una corsa tracciata in diretta;
			\item Specificare gli eventuali percorsi secondari di una linea. 
		\end{itemize}

		Anche dal punto di vista delle tecnologie utilizzate è possibile apportare delle migliorie, come illustrato di seguito. 

		\subsection{\textit{Hosting} della piattaforma}
			Per tutto lo sviluppo della piattaforma, il server è stato ospitato fisicamente sulla stessa macchina sulla quale è stato implementato. Questo implica che, al netto di un'esposizione di quella determinata macchina sulla \textit{global internet}, le API sono accessibili solamente in rete locale ma non dall'esterno. In futuro, sarebbe auspicabile utilizzare un servizio di \textbf{PaaS} (\textit{Platform as a Service}) come \textbf{Amazon AWS} per ottenere un'\textit{availability} tale da permettere il normale funzionamento della piattaforma in ogni momento. Un servizio di hosting come AWS, inoltre, permetterebbe anche di unificare l'infrastruttura della piattaforma: al momento, infatti, il server Nominatim e la Java Application che implementa la logica lato server della piattaforma risiedono su due macchine separate per ragioni logistiche. Avere le due componenti su una singola macchina consentirebbe una migliore organizzazione dell'infrastruttura.

		\subsection{Notificare gli utenti: Firebase}
			\textbf{Firebase} è una piattaforma di supporto allo sviluppo di applicazioni creata originariamente da una startup fondata dagli sviluppatori \textbf{Andrew Lee} e \textbf{James Tamplin} e successivamente acquisita da Google nel 2014. Permette di semplificare diversi task quali la gestione di database, dell'autenticazione degli utenti e delle notifiche. Quest'ultimo aspetto risulta di particolare interesse per un successivo sviluppo della piattaforma: all'utente potrebbe risultare utile ricevere delle notifiche \textit{push} ogniqualvolta siano segnalati degli eventi temporanei nella sua zona oppure nel caso sia stato assegnato ad una segnalazione di una fermata, di una linea o di un'azienda di trasporto, così da velocizzare notevolmente il processo di integrazione della segnalazione stessa all'interno di \textsc{ShallWeGo}.