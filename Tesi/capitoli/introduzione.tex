\phantomsection
%\addcontentsline{toc}{chapter}{Introduzione}
\chapter{Introduzione}
\markboth{Introduzione}{}
% [titolo ridotto se non ci dovesse stare] {titolo completo}

\section{Contesto applicativo e Motivazioni} %\label{1sec:scopo}
    \subsection{Lo stato del Trasporto Pubblico Locale}
    Negli ultimi due anni, complici anche gli avvenimenti che stanno interessando il mondo ed i successivi provvedimenti, le abitudini dei cittadini in tema di mobilità sono drasticamente cambiate.
    Tra i numerosissimi settori che sono interessati da questa ondata di cambiamenti ce n'è uno in particolare che ne ha risentito particolarmente: quello del Trasporto Pubblico Locale (\textbf{TPL}) che, secondo un rapporto stilato da varie associazioni di rappresentanza di aziende di trasporto presentato al Parlamento Italiano all'inizio del 2021 (\cite{statotpl}) e ripreso dal quotidiano "Il Sole 24 Ore" in data 25 Gennaio dello stesso anno \cite{statotplsole24ore} ha visto un crollo dei ricavi dell'entità dei Miliardi di Euro. Nello specifico, dal crollo della fruizione dei mezzi pubblici si stima che nelle casse delle aziende siano entrati complessivamente 250 milioni di Euro in meno al mese, portando le perdite complessive a circa 2 Miliardi.
    Tutto ciò, come anticipato, è causato sostanzialmente da un crollo delle presenza da parte dei cittadini su autobus, treni e metropolitane in un primo momento dovuto alle restrizioni imposte, almeno nel caso dell'Italia, nella prima metà del 2020 e complice la percezione da parte del cittadino dei mezzi pubblici come "poco sicuri" dal punto di vista sanitario risultando, continua il rapporto, in un calo del 90\% della domanda.
    Questi due fattori (crollo della domanda e conseguente diminuzione delle entrate) hanno generato in molti casi conseguenze pesanti sullo "stato di salute" delle aziende del settore, con successive rimodulazioni orarie e una sostanziale diminuzione delle corse, specie nelle prime e nelle ultime ore della giornata. La diminuzione delle corse inevitabilmente apre la porta ad uno dei problemi principali che sta affliggendo il settore nell'ultimo periodo: l'affollamento delle corse stesse.
    
    \subsection{Tecnologia e TPL}
    Con la sempre più capillare diffusione della tecnologia nelle mani del cittadino da un decennio a questa parte, i principali attori che operano in questo campo hanno riconosciuto l'importanza di andare a creare un sistema di informazione puntuale sul funzionamento del TPL in una determinata zona. Piattaforme come Google Maps, ad esempio, già da tempo forniscono dettagli sulla presenza di fermate e dettagli delle corse delle aziende di trasporto che operano in quella zona. 
    Nel caso particolare di Google Maps, questi dati, secondo la documentazione ufficiale del servizio "Google Transit" disponibile all'indirizzo \url{https://support.google.com/transitpartners/answer/1111471} sono ricavati principalmente dalle comunicazioni che giungono alla piattaforma da parte delle singole aziende che decidono di partecipare al programma, avendo poi l'opzione di condividere anche dati in tempo reale che permettano agli utenti di seguire in diretta l'andamento delle corse delle linee che gli interessano, tramite la piattaforma "Realtime Transit", che comunque è riservato alle aziende di trasporto.

    \subsection{Problemi delle attuali soluzioni}
    Come accennato in precedenza, i dati che sono disponibili sulle piattaforme più diffuse sono comunicati periodicamente dalle aziende che aderiscono a programmi come il sopracitato Google Transit su Maps. 
    Tuttavia, questo approccio non è esente da problemi di natura pratica: la presenza dei dati disponibili sulle varie piattaforme risulta quindi, nella maggior parte dei casi, essere subordinata all'adesione delle varie aziende alle stesse. Se questo problema non si pone per le aziende di medie/grandi dimensioni che hanno a disposizione risorse economiche e tecnologiche, lo stesso non si può dire per le aziende più piccole a carattere locale, che potrebbero avere difficoltà di carattere logistico anche solo nel posizionare gli adeguati segnali di riconoscimento delle fermate che utilizzano.
    Ne consegue quindi che gli utenti occasionali di queste aziende di trasporto (si pensi ad esempio ad una persona che si trova per la prima volta in una determinata zona) siano costretti a cercare informazioni presso altri cittadini che "ne sanno più di loro".
    Una piattaforma che ha provato a porre rimedio a questo problema è Moovit che ha lanciato nel 2015 un servizio chiamato "Mooviters' Community" che permette agli utenti di andare ad aggiungere dettagli alla piattaforma stessa che non siano stati già comunicati alle aziende. Questo approccio verrà trattato nel capitolo successivo dedicato proprio alle piattaforme presenti ad oggi sul mercato. 

\section{Obiettivi della tesi}
    L'obiettivo di questo lavoro è quindi proporre un primo esempio di soluzione ai due problemi di cui si è trattato in precedenza: creare una piattaforma, chiamata \textbf{ShallWeGo}, che metta a disposizione dei suoi membri gli strumenti adatti a diffondere la propria conoscenza in merito di organizzazione di linee e fermate nella loro zona con gli altri utenti, formando quindi una community che vede nella reciproca collaborazione il principale mezzo per raggiungere una vera e propria utilità per il singolo cittadino.

    
    
\section{Metodologie e risultati}
    Il risultato di questa esperienza di Tesi è stato quindi quello di aver sviluppato una prima versione di un applicativo mobile e della relativa componente server che implementasse l'idea di cui si è discusso. L'applicativo consiste quindi in un semplice file .apk per Android installabile e che, attualmente, previa registrazione ed ottenimento di uno username, permetta a chi vuole di cominciare ad effettuare segnalazioni riguardanti fermate, linee ed aziende di trasporti che conosce e che sa operare in una determinata zona.
    È, inoltre, in sviluppo una funzionalità che preveda la possibilità da parte di un utente di comunicare in tempo reale la propria posizione e, contemporaneamente, specificare su che linea si trova, così da permettere ad altri utenti interessati di farsi un'idea di quanto debbano aspettare per salire a bordo.
    \subsection{Crowdsourcing}
        La piattaforma in sviluppo basa il suo funzionamento sulla tecnica del crowdsourcing, ovvero sul recuperare i dati di interesse direttamente da persone sul campo e che ritengano di possedere un'informazione che potrebbe tornare utile alla comunità. Tra queste:
        \begin{itemize}
            \item Posizione di una fermata in un determinato punto
            \item "Equipaggiamento" di una fermata (in termini di pensilina, segni identificativi e quadri orari disponibili al pubblico)
            \item Utilizzo di una fermata da parte di una determinata linea
            \item Destinazioni di una linea
            \item Eventi transitori (come presenza di traffico, avvisi su deviazioni o strade chiuse)
        \end{itemize}

        Questo approccio permette quindi al sistema di essere indipendente da qualsivoglia fonte di informazione o API messa a disposizione pubblicamente da attori terzi (come ad esempio Google stessa o anche Moovit), i cui costi non sempre sono sostenibili sul lungo periodo.

        \subsection{Il problema delle API}
        Al di là dei dati sul trasporto pubblico locale, l'applicazione oggetto di questo lavoro di Tesi fa largo uso di mappe e dati geografici in generale. Google Maps mette a disposizione le sue API in maniera gratuita fino ad esaurimento di un credito (che viene scalato man mano che si utilizzano i propri servizi) che si rinnova ogni mese. Superata questa soglia, è necessario sostenere un certo prezzo.
        Le tariffe, consultabili all'indirizzo \url{https://cloud.google.com/maps-platform/pricing}, prevedono infatti la possibilità di usufruire un credito gratuito dell'entità di 200\$ mensili. In particolare, il servizio di Geocoding \footnote[1]{Ovvero il processo che consente di ottenere il nome logico di un luogo associato ad una coppia di coordinate (Latitudine, Longitudine) e viceversa} costa 5\$ ogni 1000 richieste. Il prezzo, quindi, aumenta all'aumentare delle richieste che vengono effettuate da chi utilizza l'applicazione.
        Sono tuttavia disponibili delle alternative di terze parti che permettono l'accesso gratuito a degli endpoint per effettuare le operazioni di Geocoding. Uno dei migliori servizi di questo genere è rappresentato da \textbf{Nominatim} che è utilizzato, tra l'altro, anche dal progetto \textbf{OpenStreetMap} per implementare la feature di ricerca nella loro piattaforma. Il funzionamento di Nominatim verrà descritto in dettaglio nel quarto capitolo della tesi che tratta l'architettura ed il funzionamento di ShallWeGo.
        
        \newpage

\section{Struttura della tesi}
        La tesi è strutturata principalmente in cinque parti, che coprono gli aspetti principali del lavoro svolto, con un'attenzione particolare alla metodologia utilizzata per implementare in modo efficiente la valutazione delle segnalazioni che pervengono alla piattaforma da parte degli utenti. \\
        In particolare:
        \begin{itemize}
            \item Il Primo Capitolo funge da introduzione al lavoro svolto per realizzare la piattaforma
            \item Il Secondo Capitolo descrive le principali soluzioni che sono attualmente utilizzabili dagli utenti per quanto riguarda il dominio del problema trattato dal lavoro di Tesi.
            \item Il Terzo Capitolo descrive accuratamente l'approccio usato per la valutazione delle segnalazioni degli utenti, che fa uso di una tecnica di \textbf{Intelligenza Artificiale} che permette di selezionare gli utenti più adatti a valutare una determinata segnalazione
            \item Il Quarto Capitolo, invece, illustra l'architettura su cui si basa l'applicazione e i dettagli implementatiti in termini di Framework, Linguaggi di Programmazione e Librerie utilizzate per lo sviluppo dell'applicativo Mobile.
            \item Infine, il Quinto Capitolo riassume tutti gli sviluppi futuri che permetterebbero all'applicazione di uscire dallo stato di "demo" ed essere quindi messa a disposizione del pubblico.
        \end{itemize}
