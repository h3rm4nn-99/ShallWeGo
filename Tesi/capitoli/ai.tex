\chapter{Background sull'Intelligenza Artificiale utilizzata} %\label{1cap:spinta_laterale}
% [titolo ridotto se non ci dovesse stare] {titolo completo}
%

\begin{citazione}
    \textit{Lo scopo di questo capitolo è di illustrare in dettaglio la componente di Intelligenza Artificiale che sta alla base del funzionamento della piattaforma.}
\end{citazione}

\newpage

\section{Descrizione del problema da affrontare}
    Come descritto in precedenza, la fonte dei dati disponibili in ShallWeGo risiede nella sua community di utenti, che mettono a disposizione la loro conoscenza su questo ambito con gli altri utenti. Il concetto base su cui si basa la piattaforma è quindi la \textbf{segnalazione} da parte del singolo utente. Di questi dati, non essendo forniti direttamente dalle aziende di trasporto, non è garantita la precisione o addirittura la correttezza. Sorge quindi il bisogno di effettuare una validazione di questi ultimi. Le persone più adatte a validare i dati di una certa segnalazione sono quelle che possiedono principalmente i seguenti requisiti:
    \begin{itemize}
        \item Risiedano (oppure operino abitualmente) in una zona vicina al luogo oggetto della segnalazione.
        \item Abbiano una certa "reputazione" all'interno della piattaforma. (ShallWeGo infatti tiene traccia del livello di attività in termini di segnalazioni e di verifica delle stesse. Questo livello, similemnet a quanto avviene sul social network Reddit, viene chiamato "\textit{karma}" ed aumenta all'aumentare dell'attività del singolo)
    \end{itemize}

    Come detto, i "verificatori" di una segnalazione sono assegnati all'atto dell'invio della stessa da parte dell'utente. Al crescere del numero di utenti della piattaforma, una ricerca esaustiva all'interno dell'insieme degli aderenti alla piattaforma risulta essere problematica in termini di complessità. Si è quindi deciso di sfruttare una tecnica particolare che permetta di evitare la ricerca esaustiva: quella degli \textbf{Algoritmi Genetici}.

\section{Algoritmi genetici}
    Un algoritmo genetico non consiste tanto in una tipologia vera e propria di algoritmi quanto un paradigma (una \textbf{metaeuristica}) che consente di implementare un algoritmo di \textbf{ricerca} per risolvere problemi di ottimizzazione. In quanto metaeuristica, il paradigma che sta alla base degli algoritmi generici non garantisce di trovare la soluzione ottima nello spazio di ricerca che si sta considerano ma permette di trovare, in maniera molto veloce (e in questo risiede uno dei suoi vantaggi che lo fanno preferire ad altri approcci) delle soluzionic he vi si avvicinano. Un altro vantaggio che rende molto utile l'impiego di algoritmi che si basano su questa metaeuristica, consiste nel fatto che essi siano sempre applicabili, a prescindere dalla struttura del problema. \\
    In particoalre, ad un algoritmo che sfrutta questa metaeuristica viene presentato un insieme iniziale di solzuioni al problema che si sta affrontano e, secondo varie metriche che forniscono una stima di quanto \textit{"buona"} sia una certa solzuione a quel determinato problema, stabilire la migliore (o le migliori), andando a crearne di nuove se necessario.

\section{Struttura e funzionamento degli algoritmi genetici}
    In generale, il funzionamento degli algoritmi genetici (e da qui la denominazione del paradigma) ricalca a grandi linee quella dell'evoluzione delle specie descritte da Charles Darwin nel 1854 nella sua opera \textit{"L'origine della specie"} per cui l'evoluzione delle specie presenti in natura segue un cammino ben preciso, che consiste in tre fasi ben precise, a partire da una popolazione iniziale fissata di individui:

    \begin{itemize}
        \item Selezione naturale
        \item Accoppiamento tra individui (detto anche, nel contesto che si sta trattando, \textit{corossover})
        \item Mutazione di un gene di un individuo (quindi una caratteristica all'interno dell'individuo cambia)
    \end{itemize}

    Un individuo, secondo quanto descritto da Darwin, può essere visto come un insieme di caratteristiche (o \textit{geni}) che ne definiscono l'identità. Mediante l'accoppiamento, i geni di due individui si mescolano, risultando nella creazione di un altro individuo che possiederà quindi una combinazione dei geni dei suoi genitori. Con una certa probabilità, inoltre, avviene il feonomeno della mutazione, descritto poc'anzi.

    Infine, sulla base di quanto un individuo possegga delle buone caratteristiche, esso potrà "sopravvivere" ed eventualmente riprodursi per formare nuovi individui o "morire", non propagando oltre i suoi geni, che evidentemente non erano abbastanza adatti all'ambiente in cui si trovava.

    La strategia utilizzata da un algoritmo che implementa questo paradigma è particolarmente simile a quella descritta da Darwin per il suo campo di studi.

    Portando avanti la similitudine con quanto avviene in natura, quando si vuole affrontare un dato problema (che chiameremo $x$) mediante la tecnica degli algoritmi genetici innanzitutto vengono generate in maniera più o meno casuale una serie di soluzioni ammissibili per $x$, che sono composte da diverse caratteristiche (quindi, da diversi \textit{geni}). Questo insieme di soluzioni al problema rappresenta la cosiddetta \textbf{popolazione iniziale}. A partire da questa popolazione viene solitamente calcolato l'indice di "bontà" di ogni singola soluzione (che in gergo viene chiamata \textbf{fitness}), tramite una funzione apposita, chiamata \textbf{\textit{funzione di Fitness}}. Una volta valutata questa quantità è necessario stabilire quanti e quali individui possano essere ammessi a riprodursi. \\
    Per questa ragione, a questo stadio si affronta solitamente la fase di \textbf{selezione}, che consiste nel far sopravvivere solamente gli individui migliori in termini di fitness. Gli individui che sopravvivono alla selezione sono ammessi al \textbf{mating pool}, ovvero saranno membri dell'insieme degli individui che si possono riprodurre. \\
    Una volta effettuato l'accoppiamento, la popolazione risultante è rappresentata da un nuovo insieme di individui che, come accennato in precedenza, posseggono una parte dei geni del primo genitore ed una parte dei geni del secondo, mutuandone quindi quelle determinate caratteristiche. \\
    Si passa all'ultimo stadio dell'evoluzione, in cui avviene (di nuovo, con una certa probabilità) la \textbf{mutazione} di uno o più geni che compongono un individuo all'interno della popolazione. \\
    A questo punto l'algoritmo si trova davanti ad una scelta: 
    \begin{itemize}
        \item Ricominciare dall'operazione di selezione usando come popolazione quella risultante dall'operazione di mutazione
        \item Terminare il processo
    \end{itemize}

    Per prendere una decisione, l'algortimo fa riferimento alla cosiddetta \textbf{condizione di terminazione}, che rappresenta le condizioni che si devono verificare per mettere fine al processo (come ad esempio il numero di iterazioni effettuate, il tempo di esecuzione o il peggioramento in termini di fitness media della popolazione dopo un numero $n$ fissato di iterazioni). Se la condizione di terminazione viene soddisfatta, allora il processo termina e viene restituita l'ultima popolazione generata.

    Le operazioni di selezione, crossover e mutazione vengono implementate tramite procedure chiamate \textbf{operatori genetici}, la cui definizione rappresenta uno degli step più importanti dell'implementazione di un algortimo di ricerca che sfrutta questa metaeuristica, assieme a stabilire come debba essere calcolata la fitness e come e quando debba terminare il processo. 

\section{L'algoritmo utilizzato da ShallWeGo}
    In questa sezione verrà descritto accuratamente l'algoritmo di ricerca utilizzato dalla piattaforma che segue il paradigma degli algoritmi genetici che va a risolvere il problema di ottimizzazione consistente nel selezionare i migliori potenziali verificatori di una determinata segnalazione.

\subsection{Codifica dell'Individuo}
    Nell'algoritmo presente in ShallWeGo, si definisce individuo un insieme di 5 utenti (nella popolazione iniziale essi sono presi in maniera casuale da quelli che operano nella provincia della segnalazione e nei comuni distanti non più di 10 km dal luogo della segnalazione se essi vivono nelle province confinanti). In questo modo, sfruttando la vicinanza territoriale, viene aumentata la possibilità che questi utenti siano a conoscenza dell'oggetto della segnalazione per la quale sono stati scelti. La distanza tra il luogo della segnalazione e quello in cui opera l'utente viene calcolata effettuando delle operazioni di geocoding tramite il server Nominatim messo a disposizione per l'occasione. Questo aspetto verrà trattato in dettaglio nel capitolo dedicato all'architettura della piattaforma.

\subsection{Funzione di Fitness}
    Come accennato, la Funzione di Fitness fornisce una stima della bontà di una certa soluzione ad un problema in termini di determinate metriche che sono per la maggior parte dei casi problem-specific.

    Nel caso di ShallWeGo sono state considerate due metriche particolari:

    \begin{itemize}
        \item La distanza in chilometri tra la posizione di una segnalazione e l'area in cui opera un utente.
        \item La "reputazione" (o karma) che un utente si è costruito durante la sua permanenza sulla piattaforma
    \end{itemize}

    Il primo aspetto è stato reputato più importante rispetto al secondo poiché in questo modo anche persone iscritte da poco alla piattaforma possano cominciare a verificare segnalazioni nell'area in cui operano. \\
    Dal punto di vista matematico, la funzione di fitness può essere formalizzata in questo modo, se $x$ rappresenta il valore (naturalmente $>=$ 0) in chilometri della distanza tra utente e luogo della segnalazione ed $y$ il valore di karma dell'utente:

    \[ f(x, y) =
  \begin{cases}
    \frac{4(250 + \frac{100}{x+0,3}) + 2y}{2}      & \quad \text{se } x \text{ < 15}\\
    \frac{4(\frac{30}{x})^2 + 2y}{2}  & \quad \text{altrimenti}
  \end{cases}
\]

La struttura dell'equazione conferma quanto accennato poc'anzi sul dare la priorità alla distanza geografica piuttosto che (almeno in un primo momento della vita della piattaforma, con una distribuzione più uniforme degli utenti) al karma. Infatti, un utente che opera in un'area che dista meno di \textbf{15km} dal luogo della segnalazione ha intuitivamente più probabilità di essere a conoscenza della struttura della rete di trasporti in quella zona (e quindi risulta più adatto al ruolo di verificatore). \\
Ed infatti, visto l'andamento della funzione $\frac{100}{x}, x \neq 0$ e vista la presenza della costante additiva (250), un utente che abita a quella distanza dalla segnalazione verrà considerato molto forte dall'algoritmo.

\section{Operatori genetici}
  Verranno ora illustrate le scelte riguardo all'implementazione dei vari operatori genetici che prendono parte al processo di evoluzione tipco degli algoritmi genetici.

    \subsection{L'operatore genetico di Selezione}
        Il primo operatore che viene applicato in una singola iterazione dell'algoritmo è quello di \textbf{Selezione}, che, riprendendo quanto descritto in precedenza, permette di stabilire a quali individui possano essere ammessi a riprodursi ed eventualmente a mutare (sulla base della loro fitness). In letteratura, esistono diversi approcci a quest'operatore. Quelli principali sono stati illustrati durante la sesta lezione del corso di \textit{Fondamenti di Intelligenza Artificiale} tenuta dal Prof. Fabio Palomba e dal Dott. Emanuele Iannone. \\
        Tra questi vanno menzionati in particolare:

        \begin{itemize}
            \item L'approccio basato su \textbf{roulette wheel} in cui gli individui con la fitness più alta hanno più possibilità di sopravvivere a questa fase, proprio come succede ad un elemento posto su una roulette che ha una porzione di area maggiore rispetto agli altri. In questo caso, la "porzione" di roulette dedicata ad un individuo è direttamente proporzionale al valore della sua fitness (normalizzata in $[0, 1]$)
            \item L'approccio basato su \textbf{rank}, nel quale ogni individuo della popolazione è ordinato decrescentemente rispetto alla sua fitness e ad ogni posizione in questa lista viene associato un numero, il cosiddetto \textit{rango}: la probabilità di selezione del singolo risulta inversamente proporzionale al suo rango.
            \item L'approccio basato su \textbf{truncation}, in qualche modo simile al precdente (ne mutua l'ordinamento degli individui) che consiste nel fissare un numero $n$ < $|P|$ (se P rappresenta la popolazione) e di selezionare i primi $n$ individui nella popolazione che hanno la fitness più alta
        \end{itemize}

        Per l'algoritmo utilizzato in ShallWeGo si è scelto di utilizzare l'approccio basato su \textbf{roulette wheel} in quanto risulta quello di più immediata comprensione ed implementazione (essendo molto più fedele a quanto avviene in natura). Inoltre, non essendoci alcun caso in cui la fitness possa scendere sotto il valore $0$ quest'approccio è stato applicabile senza problemi.

    \subsection{L'operatore genetico di Crossover}
        Dopo l'applicazione dell'operatore di Selezione (e quindi con al definizione di quelli che sono i componenti del cosiddetto \textbf{mating pool}), si procede con la fase di \textbf{crossover}, che prevede l'accoppiamento (con una determinata probabilità) di coppie di individui all'interno del mating pool. 
        Esistono diverse tecniche che permettono di implementare questa operazione. \\
        Tra queste si menzionano: \cite{}

        \begin{itemize}
            \item La tecnica \textbf{Single Point} che dati due individui x = ($x_{0}, x_{1}, ..., x_{n}$) ed y = ($y_{0}, y_{1}, ..., y_{m}$), prevede la creazione di due nuovi individui $j$ e $k$ che siano il risultato di un incrocio tra i geni dei loro genitori, dopo aver selezionato il cosiddetto \textbf{punto di taglio}, ovvero quel punto $z$ tale che:
                \begin{itemize}
                    \item $j = x_{0}, x_{1}, ..., x_{z}, y_{z+1}, ..., y_{m}$ e
                    \item $k = y_{0}, y_{1}, ..., y_{z}, x_{z+1}, ..., x_{n}$
                \end{itemize}
            \item La tecnica \textbf{\textit{k}-points}, che rappresenta una generalizzazione della tecnica Single Point, in cui un individuo viene diviso in \textit{$k$} porzioni prima di essere incrociato con un altro individuo
        \end{itemize}
        La tecnica che viene usata da ShallWeGo è quella del \textbf{Single Point}. Inoltre, onde evitare la creazione di individui troppo simili, i "genitori" già interessati da crossover non sono considerati per ulteriori accoppiamenti.
    \subsection{L'operatore genetico di Mutazione}
        L'ultimo operatore genetico che viene applicato durante un'iterazione è quello di \textbf{Mutazione} che ricalca, come accennato precedentemente, il processo di mutazione spontanea che avviene in natura. Una mutazione consiste nel cambiamento casuale di un gene all'interno di un individuo. Anche in questo caso, esistono diversi approcci che in generale possono essere seguiti. \\
        Tra questi si menzionano:
        \begin{itemize}
            \item La tecnica del \textbf{Bit Flip} che, come suggerisce il nome, è applicabile solo nel caso di individui codificati in modo \textit{binario} e consiste nel scegliere in maniera casuale un gene all'interno dell'individuo e "capovolgere" il suo valore (quindi da 0 si passa ad 1 e vice-versa).
            \item La tecnica del \textbf{Random Resetting} che prevede il cambiamento di un gene preso casualmente all'interno dell'individuo con un altro valore ammissibile per quel gene preso altrettanto casualmente.
            \item La tecnice dello \textbf{Swap} che prevede lo scambio di due geni all'interno dell'individuo
            \item La tecnica dello \textbf{Scramble} che prevede una permutazione dei geni all'interno di un individuo
        \end{itemize}

        La maggior parte di questi approcci risultano piuttosto rilevanti in un contesto in cui la posizione di un gene cambia la sua rilevanza del suo valore in un individuo, come nel caso di codifiche binarie (il bit più a sinistra se è posto ad 1 aumenta di gran lunga il valore del numero rappresentato in binario). \\
        Data la particolare natura del problema che si sta affrontando, non è possibile andare ad utilizzare la tecnica del Bit Flip e, nello stesso modo, le tecniche di Swap e Scramble non sono rilevanti in quanto la codifica dell'individuo non risente della posizione di un gene in particolare. \\
        Di conseguenza, la strategia che viene usata per questo algoritmo è quella del \textbf{Random Resetting}. In particolare, scelto un utente che è presente all'interno di un individuo, lo si sostituisce con un nuovo utente preso dal pool dei potenziali candidati ad essere verificatori di una segnalazione. Nel caso il nuovo utente che ha sostituito il precedente fosse già presente all'interno dell'individuo, il processo si ripete fino a non avere utenti uguali.
\section{Termine del processo ed accorgimenti adottati}

    \subsection{Condizione di terminazione}
        L'algoritmo termina quando si verifica una delle seguenti tre condizioni:

        \begin{itemize}
            \item Vengono superate le 25 iterazioni del processo
            \item Viene superato il limite di tempo stabilito (nel caso dell'algoritmo della piattaforma, quest'ultimo è fissato a 5 minuti)
            \item Per tre iterazioni consecutive la fitness della popolazione è minore rispetto a quella della miglior popolazione creata fino a quel momento
        \end{itemize}

        \subsection{La strategia dell'archivio}
            Come ulteriore accorgimento nello sviluppo dell'algoritmo per la piattaforma, è stata implementata la cosiddetta \textbf{Strategia dell'Archivio} che consiste nel conservare una popolazione separata che non evolve composta da individui particolarmente forti, che si va a costruire man mano che le iterazioni vanno avanti. Nel caso specifico di ShallWeGo si è provveduto, per ogni iterazione, a tenere traccia del miglior individuo in termini di fitness della popolazione risultante dall'applicazione dei tre operatori genetici. Il migliore tra questi tre verrà quindi aggiunto all'archivio.
            Essendo costituito da individui "forti", l'archivio può essere usato come alternativa alla popolazione che viene restituita dopo il termine della computazione dell'algoritmo. Questo, infatti, è proprio l'approccio scelto per la piattaforma: se al termine delle iterazioni la popoalzione risultante avrà una fitness media minore o uguale a quella dell'archivio, la popolazione restituita sarà sostituita dall'archivio. \\
            La popolazione restituita sarà quindi soggetta a delle operazioni di postprocessing.

        \subsection{Postprocessing}
            Dopo l'esecuzione dell'algoritmo il risultato è quindi una popolazione di individui, a loro volta composta da utenti. In precedenza, è stato specificato come la fitness di un individuo sia calcolata tramite una media aritmetica della fitness calcolata sui singoli geni. Si sfrutta quindi questa possibilità per effettuare un lavoro di post-processing sulla popolazione risultato. In particolare, si è scelto di effettuare le seguenti operazioni:

            \begin{itemize}
                \item Si ottiene, a partire dal singolo individuo, la lista degli utenti che esso contiene, ordinati in maniera decrescente rispetto al loro valore di fitness;
                \item Per ognuna di queste liste, viene preso il primo elemento. Il risultato di questa operazione sarà di fatto un nuovo individuo formato dagli utenti più "forti" di quelli presenti tra gli individui della popolazione risultante dall'algoritmo.
            \end{itemize}

            Sebbene le operazioni di post processing (trattandosi prettamente di ordinamento) assumano una complessità di almeno $\mathcal{O}(n\log{}n)$, esse vengono effettuate su un dominio particolarmente più piccolo rispetto a quello dei candidati (che mettendosi nel contesto di una piattaforma ben avviata potrebbero essere in numero molto elevato o comunque tale da rendere la ricerca esaustiva poco efficiente) e che allo stesso tempo permettono di ottenere un risultato ancora migliore rispetto a quello ottenuto con la semplice evoluzione, riuscendo cioè a creare un gruppo di utenti che risultino "forti tra i forti" in termini di fitness assoluta. Ciò giustifica la presenza sia della procedura di evoluzione degli individui sia il postprocessing. \\ 
            Le operazioni di ordinamento non richiedono un ulteriore calcolo della fitness in quanto in fase di implementazione è stato effettuato il caching di quel valore per quella determinata istanza dell'algoritmo e sicuramente questo valore alla fine sarà già stato calcolato e conservato, riducendo il calcolo finale della fitness ad un semplice accesso ad una variabile che non richiede operazioni particolari.


    \subsection{Risultati dell'algoritmo con i parametri correnti}
        Nella sua fase di testing, l'algoritmo è stato eseguito su un insieme molto grande di utenti generato in maniera casuale (un utente per ogni comune e con fitness presa casualmente tra 0 e 65). Nella tabella che segue sono riportati i rusultati dell'algoritmo su diverse istanze. In particolare, 

        \begin{itemize}
            \item Nella \textbf{prima colonna} è riportato il comune della segnalazione (e le sue coordinate);
            \item Nella \textbf{seconda colonna} sono riportati comune di residenza e livello di karma degli utenti selezionati dall'algoritmo  per verificare quella determinata segnalazione, dopo l'applicazione del post-processing;
        \end{itemize}

        \begin{center}

            \begin{table}[H]
                \centering
                \begin{tabular}{|l|l|}
                \hline
                Fisciano         & \begin{tabular}[c]{@{}l@{}}Cava de' Tirreni (karma: 48.6)\\ Siano (karma: 12.5)\\ Baronissi (karma: 17.8)\\ Roccapiemonte (karma: 47.8)\\ Trentinara (karma: 53.6)\end{tabular}                                          \\ \hline
                Nocera Inferiore & \begin{tabular}[c]{@{}l@{}}Nocera Inferiore (karma: 29.2)\\ Sant'Egidio del Monte Albino (karma: 39.3)\\ San Marzano sul Sarno (karma: 53.4)\\ Roccapiemonte (karma: 47.8)\\ Nocera Superiore (karma: 24.2)\end{tabular} \\ \hline
                                 &                                                                                                                                                                                                                          \\ \hline
                                 &                                                                                                                                                                                                                          \\ \hline
                \end{tabular}
                \caption{\label{tab:table-name}Risultati dell'algoritmo su diversi input.}
                \end{table}
        \end{center}
        
        

\newpage
