%\selectlanguage{italian}
\begin{abstract}
    La Tesi sviluppata si posiziona nell’ambito della creazione di strumenti informatici per assistere gli utenti del Trasporto Pubblico Locale nella loro esperienza quotidiana.
    L’obiettivo principale di questa Tesi consiste nello sviluppo di un’applicativo mobile che permetta l’accesso ad una piattaforma online chiamata \textsc{ShallWeGo} basata sulla collaborazione tra utenti che ha come scopo quello di consentire lo scambio di informazioni sull’organizzazione del Trasporto Pubblico Locale in una determinata zona. Il vero motore di \textsc{ShallWeGo} è quindi il singolo utente, che può mettere a disposizione degli altri utenti la propria conoscenza sul topic in questione (organizzazione delle fermate sul territorio, aziende di trasporto e linee espletate da queste ultime), che quindi possono sopperire alle potenziali difficoltà di comunicazione da parte delle aziende di trasporto. Particolare attenzione deve essere dedicata tuttavia all’affidabilità di queste segnalazioni che devono essere controllate in qualche modo. È stato quindi previsto, tramite la definizione di un Agente Intelligente, un sistema di verifica delle segnalazioni anch’esso basato sulla partecipazione attiva degli utenti della piattaforma. Data una segnalazione sarà quindi possibile andare a determinare il gruppo di utenti che secondo determinate “metriche” risultino più adatti a verificarla e stabilire se quest'ultima possa essere integrata o meno sulla piattaforma.
\\[1cm]
\end{abstract} 